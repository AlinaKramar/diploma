\documentclass{beamer}
\usepackage{fontspec}
\usepackage{fontspec}
\usepackage{xunicode}
\usepackage{caption}
\usepackage{xltxtra}
\usepackage{xecyr}
\usepackage{hyperref}
\setmainfont[Mapping=tex-text]{DejaVu Serif}
\setsansfont[Mapping=tex-text]{DejaVu Sans}
\setmonofont[Mapping=tex-text]{DejaVu Sans Mono}
\usepackage{polyglossia}
\setdefaultlanguage{russian}
\usepackage{graphicx}
\usepackage{listings}

\mode<presentation> {
\usetheme{Singapore}
\usecolortheme{dove}

\setbeamertemplate{footline}[page number]
%\setbeamertemplate{caption}[numbered]
\setbeamertemplate{caption}{\insertcaption}
}

\usepackage{graphicx}
\usepackage{booktabs}

\title{Построение генетических карт по неполным и
зашумленным данным}
\author{Алина Крамар\\
  { \footnotesize \textcolor{gray}{группа 545\\ руководитель Сысоев
      С.С\\ рецензент Добрынин П.B.}}}
\institute{Санкт-Петербургский государственный университет}
\date{\today}

\begin{document}

\begin{frame}
\titlepage
\end{frame}


\begin{frame}
  \frametitle{Необходимые определения}

\end{frame}

\begin{frame}
  \frametitle{Генетические карты}
  \begin{itemize}
    \item При получении генетического материала невозможно дать ответ
      о расположении маркеров на хромосоме
    \item Кроссинговер вносит возмущение в этот порядок
    \item Порядок маркеров и формирование гамет происходит во время мейоза
  \end{itemize}
\end{frame}

\begin{frame}
  \frametitle{Существующие решения}
  \begin{itemize}
      \item CRIMAP

        (Экспоненциальная сложность от количества особей в родословной)
      \item FASTLINK

        (Экспоненциальная сложность от количества маркеров)
      \item genmap

        (Степенная сложность от количества особей и маркеров)
  \end{itemize}
\end{frame}

\begin{frame}
  \frametitle{Постановка задачи}
  Написать новый алгоритм генетического картирования на базе метода
  прямого извлечения данных (genmap)

  \medskip

  Задачи:
  \begin{itemize}
  \item Доработка алгоритма прямого извлечения данных
  \item Верификация полученного алгоритма
    \begin{itemize}
      \item моделирование и генерация тестовых данных
      \item сравнение существующих алгоритмов с полученной версией
    \end{itemize}
  \end{itemize}
\end{frame}

\begin{frame}
  \frametitle{Недостатки genmap}
  \begin{itemize}
    \item Неустойчивость к вариациям входных данных
    \item Неоптимальная реализация
    \item Неправильный результат в случае кратного кроссинговера
  \end{itemize}
\end{frame}

\begin{frame}
  \frametitle{Генерация тестовых данных}
  \begin{itemize}
    \item Стратегия кроссовера
    \item Численные параметры родословной
    \item Симуляция ошибок секвенирования
  \end{itemize}
\end{frame}

\begin{frame}
  \frametitle{Пример родословной}

\end{frame}

\begin{frame}
  \frametitle{Результаты сравнения алгоритмов}
 % +----------------+---------------+---------------+---------------+
 % |                |CRIMAP         |FASTLINK       |Genmap2        |
 % |Особи/Маркеры   |               |               |               |
 % +----------------+---------------+---------------+---------------+
 % |200/30          |0.3 секунд     |10 минут       |1 час          |
 % |                |               |               |               |
 % +----------------+---------------+---------------+---------------+
 % |200/50          |0.3 секунды    |10 минут       |\infty         |
 % |                |               |               |               |
 % +----------------+---------------+---------------+---------------+
 % |4000/40         |6 секунд       |\infty         |3 дня          |
 % |                |               |               |               |
 % +----------------+---------------+---------------+---------------+


\begin{tabular}{|l|l|l|l|}
  \hline
  & CRIMAP & FASTLINK & Genmap2 \\
  Особи/Маркеры & & & \\
  \hline
  200/30 & 0.3 секунд & 10 минут & 1 час \\
  & & & \\
  \hline
  200/50 & 0.3 секунды & 10 минут & $ \infty $ \\
  & & & \\
  \hline
  4000/40 & 6 секунд & $ \infty $ & 3 дня \\
  & & & \\
  \hline
\end{tabular}

\end{frame}


\begin{frame}
  \frametitle{Учёт кратности кроссинговера}

\end{frame}

\begin{frame}
  \frametitle{Результаты}
  \begin{itemize}
  \item Реализован более хороший алгоритм на основе genmap
  \item Проведена его верификация
    \begin{itemize}
    \item путём моделирования и генерации тестовых данных
    \item экспериментально показав его сильные стороны
    \end{itemize}
  \item Выступление на конференции СПИСОК-2014
  \end{itemize}
\end{frame}

\end{document}

%%% Local Variables:
%%% coding: utf-8
%%% mode: latex
%%% TeX-engine: xetex
%%% End:
