\documentclass{beamer}
\usepackage{fontspec}
\usepackage{fontspec}
\usepackage{xunicode}
\usepackage{caption}
\usepackage{xltxtra}
\usepackage{xecyr}
\usepackage{hyperref}
\setmainfont[Mapping=tex-text]{DejaVu Serif}
\setsansfont[Mapping=tex-text]{DejaVu Sans}
\setmonofont[Mapping=tex-text]{DejaVu Sans Mono}
\usepackage{polyglossia}
\setdefaultlanguage{russian}
\usepackage{graphicx}
\usepackage{listings}

\mode<presentation> {
\usetheme{Singapore}
\usecolortheme{dove}

\setbeamertemplate{footline}[page number]
%\setbeamertemplate{caption}[numbered]
\setbeamertemplate{caption}{\insertcaption}
}

\usepackage{graphicx}
\usepackage{booktabs}

\title{Построение генетических карт по неполным и
зашумленным данным}
\author{Алина Крамар\\
  { \footnotesize \textcolor{gray}{группа 545\\ руководитель Сысоев
      С.С\\ рецензент Добрынин П.}}}
\institute{Санкт-Петербургский государственный университет}
\date{\today}

\begin{document}

\begin{frame}
\titlepage
\end{frame}


\begin{frame}
  \frametitle{Необходимые определения}

\end{frame}

\begin{frame}
  \frametitle{Генетические карты}
  \begin{itemize}
    \item При получении генетического материала невозможно дать ответ
      о расположении маркеров на хромосоме
    \item Кроссинговер вносит возмущение в этот порядок
    \item Порядок маркеров и формирование гамет происходит во время мейоза
  \end{itemize}
\end{frame}

\begin{frame}
  \frametitle{Существующие решения}
  \begin{itemize}
      \item CRIMAP

        (Экспоненциальная сложность от количества особей в родословной)
      \item LM\_MAP

        (Экспоненциальная сложность от количества маркеров)
      \item Прототип алгоритма Сергея Сысоева

        (Степенная сложность от количества особей и маркеров)
  \end{itemize}
\end{frame}

\begin{frame}
  \frametitle{Постановка задачи}
  Сравнить существующие алгоритмы генетического картирования и
  улучшить алгоритм Сергея Сысоева

  Задачи:
  \begin{itemize}
  \item Сравнение алгоритмов генетического картирования
    \begin{itemize}
      \item реализовать алгоритм генерации тестовых данных
      \item сравнить существующие алгоритмы на реализованной базе
    \end{itemize}
  \item Усовершенствование алгоритма Сысоева
    \begin{itemize}
      \item выявить недостатки алгоритма путём
      \item реализовать новую версию алгоритма
      \item выложить в открытый доступ новую версию
    \end{itemize}
  \end{itemize}
\end{frame}

\begin{frame}
  \frametitle{Генерация тестовых данных}
  \begin{itemize}
    \item Моделирование мейоза и скрещивания
    \item Генерация родословных по заданным параметрам
    \item Симуляция мутаций
    \item Генерация несоответствующих модели данных
  \end{itemize}
\end{frame}

\begin{frame}
  \frametitle{Пример родословной}

\end{frame}

\begin{frame}
  \frametitle{Результаты сравнения алгоритмов}
 % +----------------+---------------+---------------+---------------+
 % |                |CRIMAP         |LM\_MAP        |genmap.py      |
 % |Особи/Маркеры   |               |               |               |
 % +----------------+---------------+---------------+---------------+
 % |192/35          |120 секунд     |111 секунд     |17 секунд      |
 % |                |               |               |               |
 % +----------------+---------------+---------------+---------------+
 % |2011/146        |1 день         |1 день         |11 минут       |
 % |                |               |               |               |
 % +----------------+---------------+---------------+---------------+
 % |100000/100      |10+ дней       |9+ дней        |37 минут       |
 % |                |               |               |               |
 % +----------------+---------------+---------------+---------------+


  \begin{tabular}{|l|l|l|l|}
    \hline
    & CRIMAP & LM$\backslash$\_MAP & genmap.py \\
    Особи/Маркеры & & & \\
    \hline
    192/35 & 120 секунд & 111 секунд & 17 секунд \\
    & & & \\
    \hline
    2011/146 & 1 день & 1 день & 11 минут \\
    & & & \\
    \hline
    100000/100 & 10+ дней & 9+ дней & 37 минут \\
    & & & \\
    \hline
  \end{tabular}


\end{frame}

\begin{frame}
  \frametitle{Результаты тестирования алгоритма С.Сысоева}
  \begin{itemize}
    \item Неустойчивость к вариациям входных данных
    \item Неоптимальная реализация
    \item Неправильный результат в случае кратного кроссинговера
  \end{itemize}
\end{frame}

\begin{frame}
  \frametitle{Учёт кратности кроссинговера}
\end{frame}

\begin{frame}
  \frametitle{Результаты}
  \begin{itemize}
  \item Произведено сравнение алгоритмов
    \begin{itemize}
      \item реализован алгоритм генерации тестовых данных
      \item произведено сравнение алгоритмов с помощью реализованной базы
    \end{itemize}
  \item Улучшения алгоритма С.Сысоева
    \begin{itemize}
      \item произведено тестирование алгоритма на синтетических данных
      \item учёт кратных рекомбинаций
      \item скрипт выложен в открытый доступ (URL)
    \end{itemize}
  \item Выступление на конференции СПИСОК-2014
  \end{itemize}
\end{frame}

\end{document}